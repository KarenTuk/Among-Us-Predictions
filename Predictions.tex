% Options for packages loaded elsewhere
\PassOptionsToPackage{unicode}{hyperref}
\PassOptionsToPackage{hyphens}{url}
%
\documentclass[
]{article}
\usepackage{amsmath,amssymb}
\usepackage{lmodern}
\usepackage{ifxetex,ifluatex}
\ifnum 0\ifxetex 1\fi\ifluatex 1\fi=0 % if pdftex
  \usepackage[T1]{fontenc}
  \usepackage[utf8]{inputenc}
  \usepackage{textcomp} % provide euro and other symbols
\else % if luatex or xetex
  \usepackage{unicode-math}
  \defaultfontfeatures{Scale=MatchLowercase}
  \defaultfontfeatures[\rmfamily]{Ligatures=TeX,Scale=1}
\fi
% Use upquote if available, for straight quotes in verbatim environments
\IfFileExists{upquote.sty}{\usepackage{upquote}}{}
\IfFileExists{microtype.sty}{% use microtype if available
  \usepackage[]{microtype}
  \UseMicrotypeSet[protrusion]{basicmath} % disable protrusion for tt fonts
}{}
\makeatletter
\@ifundefined{KOMAClassName}{% if non-KOMA class
  \IfFileExists{parskip.sty}{%
    \usepackage{parskip}
  }{% else
    \setlength{\parindent}{0pt}
    \setlength{\parskip}{6pt plus 2pt minus 1pt}}
}{% if KOMA class
  \KOMAoptions{parskip=half}}
\makeatother
\usepackage{xcolor}
\IfFileExists{xurl.sty}{\usepackage{xurl}}{} % add URL line breaks if available
\IfFileExists{bookmark.sty}{\usepackage{bookmark}}{\usepackage{hyperref}}
\hypersetup{
  pdftitle={Predictions},
  hidelinks,
  pdfcreator={LaTeX via pandoc}}
\urlstyle{same} % disable monospaced font for URLs
\usepackage[margin=1in]{geometry}
\usepackage{color}
\usepackage{fancyvrb}
\newcommand{\VerbBar}{|}
\newcommand{\VERB}{\Verb[commandchars=\\\{\}]}
\DefineVerbatimEnvironment{Highlighting}{Verbatim}{commandchars=\\\{\}}
% Add ',fontsize=\small' for more characters per line
\usepackage{framed}
\definecolor{shadecolor}{RGB}{248,248,248}
\newenvironment{Shaded}{\begin{snugshade}}{\end{snugshade}}
\newcommand{\AlertTok}[1]{\textcolor[rgb]{0.94,0.16,0.16}{#1}}
\newcommand{\AnnotationTok}[1]{\textcolor[rgb]{0.56,0.35,0.01}{\textbf{\textit{#1}}}}
\newcommand{\AttributeTok}[1]{\textcolor[rgb]{0.77,0.63,0.00}{#1}}
\newcommand{\BaseNTok}[1]{\textcolor[rgb]{0.00,0.00,0.81}{#1}}
\newcommand{\BuiltInTok}[1]{#1}
\newcommand{\CharTok}[1]{\textcolor[rgb]{0.31,0.60,0.02}{#1}}
\newcommand{\CommentTok}[1]{\textcolor[rgb]{0.56,0.35,0.01}{\textit{#1}}}
\newcommand{\CommentVarTok}[1]{\textcolor[rgb]{0.56,0.35,0.01}{\textbf{\textit{#1}}}}
\newcommand{\ConstantTok}[1]{\textcolor[rgb]{0.00,0.00,0.00}{#1}}
\newcommand{\ControlFlowTok}[1]{\textcolor[rgb]{0.13,0.29,0.53}{\textbf{#1}}}
\newcommand{\DataTypeTok}[1]{\textcolor[rgb]{0.13,0.29,0.53}{#1}}
\newcommand{\DecValTok}[1]{\textcolor[rgb]{0.00,0.00,0.81}{#1}}
\newcommand{\DocumentationTok}[1]{\textcolor[rgb]{0.56,0.35,0.01}{\textbf{\textit{#1}}}}
\newcommand{\ErrorTok}[1]{\textcolor[rgb]{0.64,0.00,0.00}{\textbf{#1}}}
\newcommand{\ExtensionTok}[1]{#1}
\newcommand{\FloatTok}[1]{\textcolor[rgb]{0.00,0.00,0.81}{#1}}
\newcommand{\FunctionTok}[1]{\textcolor[rgb]{0.00,0.00,0.00}{#1}}
\newcommand{\ImportTok}[1]{#1}
\newcommand{\InformationTok}[1]{\textcolor[rgb]{0.56,0.35,0.01}{\textbf{\textit{#1}}}}
\newcommand{\KeywordTok}[1]{\textcolor[rgb]{0.13,0.29,0.53}{\textbf{#1}}}
\newcommand{\NormalTok}[1]{#1}
\newcommand{\OperatorTok}[1]{\textcolor[rgb]{0.81,0.36,0.00}{\textbf{#1}}}
\newcommand{\OtherTok}[1]{\textcolor[rgb]{0.56,0.35,0.01}{#1}}
\newcommand{\PreprocessorTok}[1]{\textcolor[rgb]{0.56,0.35,0.01}{\textit{#1}}}
\newcommand{\RegionMarkerTok}[1]{#1}
\newcommand{\SpecialCharTok}[1]{\textcolor[rgb]{0.00,0.00,0.00}{#1}}
\newcommand{\SpecialStringTok}[1]{\textcolor[rgb]{0.31,0.60,0.02}{#1}}
\newcommand{\StringTok}[1]{\textcolor[rgb]{0.31,0.60,0.02}{#1}}
\newcommand{\VariableTok}[1]{\textcolor[rgb]{0.00,0.00,0.00}{#1}}
\newcommand{\VerbatimStringTok}[1]{\textcolor[rgb]{0.31,0.60,0.02}{#1}}
\newcommand{\WarningTok}[1]{\textcolor[rgb]{0.56,0.35,0.01}{\textbf{\textit{#1}}}}
\usepackage{graphicx}
\makeatletter
\def\maxwidth{\ifdim\Gin@nat@width>\linewidth\linewidth\else\Gin@nat@width\fi}
\def\maxheight{\ifdim\Gin@nat@height>\textheight\textheight\else\Gin@nat@height\fi}
\makeatother
% Scale images if necessary, so that they will not overflow the page
% margins by default, and it is still possible to overwrite the defaults
% using explicit options in \includegraphics[width, height, ...]{}
\setkeys{Gin}{width=\maxwidth,height=\maxheight,keepaspectratio}
% Set default figure placement to htbp
\makeatletter
\def\fps@figure{htbp}
\makeatother
\setlength{\emergencystretch}{3em} % prevent overfull lines
\providecommand{\tightlist}{%
  \setlength{\itemsep}{0pt}\setlength{\parskip}{0pt}}
\setcounter{secnumdepth}{-\maxdimen} % remove section numbering
\ifluatex
  \usepackage{selnolig}  % disable illegal ligatures
\fi

\title{Predictions}
\author{}
\date{\vspace{-2.5em}}

\begin{document}
\maketitle

\hypertarget{r-markdown}{%
\subsection{R Markdown}\label{r-markdown}}

This is an R Markdown document. Markdown is a simple formatting syntax
for authoring HTML, PDF, and MS Word documents. For more details on
using R Markdown see \url{http://rmarkdown.rstudio.com}.

When you click the \textbf{Knit} button a document will be generated
that includes both content as well as the output of any embedded R code
chunks within the document. You can embed an R code chunk like this:

\begin{Shaded}
\begin{Highlighting}[]
\CommentTok{\# Loading in the tidyverse library and the Among Us Game Data}
\FunctionTok{library}\NormalTok{(tidyverse)}
\end{Highlighting}
\end{Shaded}

\begin{verbatim}
## -- Attaching packages --------------------------------------- tidyverse 1.3.1 --
\end{verbatim}

\begin{verbatim}
## v ggplot2 3.3.5     v purrr   0.3.4
## v tibble  3.1.6     v dplyr   1.0.7
## v tidyr   1.1.4     v stringr 1.4.0
## v readr   2.1.1     v forcats 0.5.1
\end{verbatim}

\begin{verbatim}
## -- Conflicts ------------------------------------------ tidyverse_conflicts() --
## x dplyr::filter() masks stats::filter()
## x dplyr::lag()    masks stats::lag()
\end{verbatim}

\begin{Shaded}
\begin{Highlighting}[]
\NormalTok{amongus }\OtherTok{\textless{}{-}}\NormalTok{ readr}\SpecialCharTok{::}\FunctionTok{read\_csv}\NormalTok{(}\StringTok{"User1.csv"}\NormalTok{)}
\end{Highlighting}
\end{Shaded}

\begin{verbatim}
## Rows: 100 Columns: 13
\end{verbatim}

\begin{verbatim}
## -- Column specification --------------------------------------------------------
## Delimiter: ","
## chr (13): Game Completed Date, Team, Outcome, Task Completed, All Tasks Comp...
\end{verbatim}

\begin{verbatim}
## 
## i Use `spec()` to retrieve the full column specification for this data.
## i Specify the column types or set `show_col_types = FALSE` to quiet this message.
\end{verbatim}

\begin{Shaded}
\begin{Highlighting}[]
\NormalTok{amongus }\SpecialCharTok{\%\textgreater{}\%} \FunctionTok{glimpse}\NormalTok{()}
\end{Highlighting}
\end{Shaded}

\begin{verbatim}
## Rows: 100
## Columns: 13
## $ `Game Completed Date`        <chr> "12/13/2020 at 1:26:56 am EST", "12/13/20~
## $ Team                         <chr> "Crewmate", "Crewmate", "Crewmate", "Impo~
## $ Outcome                      <chr> "Win", "Loss", "Win", "Win", "Loss", "Los~
## $ `Task Completed`             <chr> "3", "7", "3", "-", "4", "7", "8", "8", "~
## $ `All Tasks Completed`        <chr> "No", "Yes", "No", "-", "No", "Yes", "Yes~
## $ Murdered                     <chr> "Yes", "No", "No", "-", "No", "Yes", "Yes~
## $ `Imposter Kills`             <chr> "-", "-", "-", "2", "-", "-", "-", "-", "~
## $ `Game Length`                <chr> "07m 04s", "16m 21s", "11m 33s", "08m 05s~
## $ Ejected                      <chr> "No", "No", "No", "No", "No", "No", "No",~
## $ `Sabotages Fixed`            <chr> "2", "1", "0", "N/A", "0", "0", "1", "0",~
## $ `Time to complete all tasks` <chr> "-", "09m 48s", "-", "-", "-", "12m 16s",~
## $ `Rank Change`                <chr> "++", "--", "++", "+++", "---", "--", "--~
## $ `Region/Game Code`           <chr> "NA / WYMSBF", "NA / WYMSBF", "NA / WYMSB~
\end{verbatim}

\begin{Shaded}
\begin{Highlighting}[]
\CommentTok{\# Retrieving important values from the data that will help us analyze and make predictions}
\CommentTok{\# Determining the total number of games and the number of wins and losses}

\NormalTok{total }\OtherTok{\textless{}{-}}\NormalTok{ amongus }\SpecialCharTok{\%\textgreater{}\%} \FunctionTok{count}\NormalTok{()}
\NormalTok{total[[}\DecValTok{1}\NormalTok{]]}
\end{Highlighting}
\end{Shaded}

\begin{verbatim}
## [1] 100
\end{verbatim}

\begin{Shaded}
\begin{Highlighting}[]
\NormalTok{Outcome }\OtherTok{\textless{}{-}}\NormalTok{ amongus }\SpecialCharTok{\%\textgreater{}\%} \FunctionTok{group\_by}\NormalTok{(Outcome)}
\NormalTok{mogus\_tibble }\OtherTok{\textless{}{-}} \FunctionTok{tibble}\NormalTok{(Outcome)}
\NormalTok{mogus\_tibble }\SpecialCharTok{\%\textgreater{}\%} \FunctionTok{group\_by}\NormalTok{(Outcome) }\SpecialCharTok{\%\textgreater{}\%} \FunctionTok{summarize}\NormalTok{(}\AttributeTok{count=}\FunctionTok{n}\NormalTok{())}
\end{Highlighting}
\end{Shaded}

\begin{verbatim}
## # A tibble: 2 x 2
##   Outcome count
##   <chr>   <int>
## 1 Loss       42
## 2 Win        58
\end{verbatim}

\begin{Shaded}
\begin{Highlighting}[]
\CommentTok{\# Further visualizing the predictions}
\CommentTok{\# Using the sample function to select random samples from the data and find similar probabilities for winning a game of Among Us based on the data}

\NormalTok{mogus\_tibble }\SpecialCharTok{\%\textgreater{}\%}
  \FunctionTok{sample\_n}\NormalTok{(}\AttributeTok{size =}\NormalTok{ total[[}\DecValTok{1}\NormalTok{]], }\AttributeTok{replace =} \ConstantTok{TRUE}\NormalTok{) }\SpecialCharTok{\%\textgreater{}\%}
  \FunctionTok{group\_by}\NormalTok{(Outcome) }\SpecialCharTok{\%\textgreater{}\%}
  \FunctionTok{filter}\NormalTok{(Outcome }\SpecialCharTok{==} \StringTok{"Win"}\NormalTok{) }\SpecialCharTok{\%\textgreater{}\%}
  \FunctionTok{summarize}\NormalTok{(}\FunctionTok{n}\NormalTok{(), }\AttributeTok{p\_win =} \FunctionTok{n}\NormalTok{()}\SpecialCharTok{/}\NormalTok{total[[}\DecValTok{1}\NormalTok{]])}
\end{Highlighting}
\end{Shaded}

\begin{verbatim}
## # A tibble: 1 x 3
##   Outcome `n()` p_win
##   <chr>   <int> <dbl>
## 1 Win        60   0.6
\end{verbatim}

\begin{Shaded}
\begin{Highlighting}[]
\CommentTok{\# Estimating the sampling distribution of probability to win the next game of Among Us based on the data (wins/losses of previous games)}
\CommentTok{\# The sampling distribution is the distribution of all values that the probability of winning a game could be for random samples from the data}
\CommentTok{\# To estimate the sampling distribution, I took 1000 values from the probabilty of winning and calculated 1000 randoms samples from the population (main data)}

\NormalTok{sample\_pwin }\OtherTok{\textless{}{-}} \FunctionTok{rep}\NormalTok{(}\ConstantTok{NA}\NormalTok{, }\DecValTok{1000}\NormalTok{)}

\ControlFlowTok{for}\NormalTok{(i }\ControlFlowTok{in} \DecValTok{1}\SpecialCharTok{:}\DecValTok{1000}\NormalTok{)\{}
\NormalTok{  sample\_100 }\OtherTok{\textless{}{-}}\NormalTok{ mogus\_tibble }\SpecialCharTok{\%\textgreater{}\%} \FunctionTok{sample\_n}\NormalTok{(}\AttributeTok{size=}\NormalTok{total[[}\DecValTok{1}\NormalTok{]], }\AttributeTok{replace =} \ConstantTok{TRUE}\NormalTok{)}
\NormalTok{  sample\_pwin[i] }\OtherTok{\textless{}{-}}
\NormalTok{    sample\_100 }\SpecialCharTok{\%\textgreater{}\%}
    \FunctionTok{filter}\NormalTok{(Outcome }\SpecialCharTok{==} \StringTok{"Win"}\NormalTok{) }\SpecialCharTok{\%\textgreater{}\%}
    \FunctionTok{summarise}\NormalTok{(}\FunctionTok{n}\NormalTok{()}\SpecialCharTok{/}\NormalTok{total[[}\DecValTok{1}\NormalTok{]]) }\SpecialCharTok{\%\textgreater{}\%}
    \FunctionTok{as.numeric}\NormalTok{()}
\NormalTok{\}}

\NormalTok{sample\_pwin }\OtherTok{\textless{}{-}} \FunctionTok{tibble}\NormalTok{(}\AttributeTok{p\_win =}\NormalTok{ (sample\_pwin)) }\SpecialCharTok{\%\textgreater{}\%} \FunctionTok{glimpse}\NormalTok{()}
\end{Highlighting}
\end{Shaded}

\begin{verbatim}
## Rows: 1,000
## Columns: 1
## $ p_win <dbl> 0.58, 0.57, 0.60, 0.55, 0.55, 0.57, 0.47, 0.59, 0.65, 0.59, 0.54~
\end{verbatim}

\begin{Shaded}
\begin{Highlighting}[]
\CommentTok{\# This is a histogram displaying the sampling distribution of the probability of winning a game of Among Us}
\CommentTok{\# The graph is normal distribution }
\CommentTok{\# Also calculated the confidence interval of the sampling distribution, where we can estimate with a 95\% confidence what the probability of winning the next game of Among Us}

\FunctionTok{ggplot}\NormalTok{(sample\_pwin, }\FunctionTok{aes}\NormalTok{(}\AttributeTok{x =}\NormalTok{ p\_win)) }\SpecialCharTok{+} \FunctionTok{geom\_histogram}\NormalTok{()}
\end{Highlighting}
\end{Shaded}

\begin{verbatim}
## `stat_bin()` using `bins = 30`. Pick better value with `binwidth`.
\end{verbatim}

\includegraphics{Predictions_files/figure-latex/unnamed-chunk-5-1.pdf}

\begin{Shaded}
\begin{Highlighting}[]
\NormalTok{quantile\_tibble }\OtherTok{\textless{}{-}} \FunctionTok{quantile}\NormalTok{(sample\_pwin}\SpecialCharTok{$}\NormalTok{p\_win, }\FunctionTok{c}\NormalTok{(}\FloatTok{0.025}\NormalTok{, }\FloatTok{0.975}\NormalTok{))}
\NormalTok{quantile\_tibble}
\end{Highlighting}
\end{Shaded}

\begin{verbatim}
##  2.5% 97.5% 
##  0.48  0.68
\end{verbatim}

\end{document}
